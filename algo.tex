\documentclass{article}
\usepackage{fullpage}
\usepackage[utf8x]{inputenc}
\usepackage{amssymb}
\usepackage{amsthm}
\usepackage{amsmath}
\usepackage{hyperref}
\usepackage{cleveref}
\usepackage{autonum}
\usepackage{dsfont}
\usepackage{stmaryrd}

\newtheorem{theorem}{Theorem}
\newtheorem{lemma}{Lemma}
\newtheorem{remark}{Remark}
\newtheorem{corollary}{Corollary}


\def\1{{\mathds 1}}
\def\N{\mathbb N}
\def\R{\mathbb R}
\def\G{{\cal G}}
\newcommand{\intset}[1]{\llbracket #1 \rrbracket}
\newcommand{\intpart}[1]{\lceil #1 \rceil}

\begin{document}

\noindent Bruno Scherrer, bruno.scherrer@inria.fr, \today.

~\\

Consider a parity game $G$ between EVEN (or Player 0) and ODD (or Player 1) on a state space $X=\{1,\dots,n\}$ with priority function $\Omega:X \to \{1,\dots,d\}$. On infinite plays, ODD wants that among the priorities that appears infinitely often, the maximal one is odd, while EVEN wants it to be even.

Let us make the game a bit more precise: a player that wins wants to do so with the highest possible priority, while a player that loses wants to do so with the lowest possible priority. This specification can for instance be dealt by Puri's reduction to a mean-payoff game. We shall call the function that assigns the resulting priority to each starting state the optimal parity function $p_*:X \to \{1,\dots,d\}$. 

Informally, we consider the following (recursive) algorithm that takes as input a game $\G$ of which the higher priority is $p$ corresponding to Player $i=p \mod 2$, and return the optimal parity $p_*:X \to \{1,\dots,d\}$.
\begin{enumerate}
\item In the game $\G$, compute the region $A$  that can be won with priority $p$ by Player $i$ and the region $B$ where Player $1-i$ can be sure that the hight priority occurring infinitely often is smaller than $p$.
\item Consider the game $\G'$ reduced to $B$. By definition of $B$, for any stationary policy of Player $i$, there exists a stationary policy for Player $1-i$ such that after at most $n$ steps, the play only visit states with priority smaller that $p$. Compute the region $C$ from which whatever Player $i$ does, the play cannot reach a state with priority $p$ (to play optimally, Player $1-i$ must force the play to reach this region).
\item Recursively solve the game $\G''$ reduced to $C$ (it only involves states with parity smaller than $p$) and obtain its optimal parity function $q_*:C \to \{1,\dots,p-1\}$. 
\item On the game $G'$, compute the optimal priority function $r_*$ by propagating the parity function from $C$ to $B$: among the policies for Player $1-i$ that prevent Player $i$ from getting to any state of $B \backslash C$, compute the one that reaches a state $y \in C$ with the best priority $q_*(y)$ (against the best adversary played by Player $i$). 
\item Return the optimal parity $p_*$ where $p_*(x)=p$ if $x \in A$ and $p_*(x)=r_*(x)$ if $x \in B$. 
\end{enumerate}

Let us a give a more formal description. The sets $A$, $B$ and $C$ (steps 1 and 2) can be computed as follows. Consider the cost function
\begin{align}
  g(x) & = \left\{
  \begin{array}{ll}
    (-1)^i & \mbox{if }\Omega(x)=p, \\
    0 & \mbox{if }\Omega(x)<p.
  \end{array}
  \right.
\end{align}
Compute the solution of the $n^2$-horizon control problem with cost $g$ and terminal cost $0$, i.e. compute $v=T^{n^2}0$ where the operator $T$ is defined as follows for any $w$:
\begin{align}
  T w(x)= \left\{
  \begin{array}{ll}
    g(x) + max_{y \in \succ{x}} w(y)& \mbox{if $x$ is controlled by Player $i$} \\
    g(x) + min_{y \in \succ{x}} w(y)& \mbox{if $x$ is controlled by Player $1-i$}
  \end{array}
  \right.
\end{align}
Then
\begin{align}
  A & = \{ x ~;~ |v(x)| \ge n \} \\
  B & = \{ x ~;~ |v(x)| < n \} \\
  C & = \{ x ~;~ |v(x)| = 0 \}.
\end{align}
For the propagation (step 4), we consider the following terminal cost function
\begin{align}
  h(x) =  \left\{
  \begin{array}{ll}
    q_*(x) & \mbox{if }x \in C, \\
    (-1)^i \infty & \mbox{if }x \in B \backslash C.
  \end{array}
  \right.
\end{align}
In this terminal cost function, the values on $B$ are chosen to force Player $1-i$ to eventually terminate to any state of $B \backslash C$ (i.e. to be sure to terminate to some state of $C$).
We finally compute $r_* = U^n h$ where the operator $U$ is defined as follows for any $w$:
\begin{align}
  U w(x)= \left\{
  \begin{array}{ll}
    max_{y \in \succ{x}} w(y)& \mbox{if $x$ is controlled by Player $i$} \\
    min_{y \in \succ{x}} w(y)& \mbox{if $x$ is controlled by Player $1-i$}.
  \end{array}
  \right.
\end{align}

\bibliographystyle{plain}
\bibliography{biblio.bib} 

\end{document}
