\documentclass{article}
\usepackage{fullpage}
\usepackage[utf8x]{inputenc}
\usepackage{amssymb}
\usepackage{amsthm}
\usepackage{amsmath}
\usepackage{hyperref}
\usepackage{cleveref}
\usepackage{autonum}
\usepackage{dsfont}
\usepackage{stmaryrd}

\newtheorem{theorem}{Theorem}
\newtheorem{lemma}{Lemma}
\newtheorem{remark}{Remark}
\newtheorem{corollary}{Corollary}


\def\1{{\mathds 1}}
\def\G{{\cal G}}
\def\C{{\cal C}}
\def\V{{\cal V}}
\def\Ent{\mathbb N}
\def\R{\mathbb R}
\def\N{\mathfrak N}
\def\M{\mathfrak M}
\def\greed{\mathfrak G}
\newcommand{\suc}[1]{f(#1)}

\def\={\stackrel{def}{=}}
\newcommand{\intset}[1]{\llbracket #1 \rrbracket}
\newcommand{\intpart}[1]{\lceil #1 \rceil}

\begin{document}

Consider a parity game $G$ on state space $X=\{1,\dots,n\}$ with priority function $\Omega:X \to \{1,\dots,d\}$.

\paragraph{Algo 1}

Consider the following algorithm that takes a game as input and update $W$
\begin{enumerate}
\item Let $p$ be the maximal priority
\item Let $i=p \mod 2$ be the corresponding player.
\item Let
  \begin{align}
    g(x) = (-1)^{\Omega(x)} \1_{\Omega(x)=p}
  \end{align}
  and
  \begin{align}
    T v = g + \max_\mu \min_\nu P_{\mu,\nu} v.
  \end{align}
  \item Compute $v=T^n 0$ 
  \item Let $A = \{ x ; T^n0(x)=0 \}$
  \item Let $B = (1-i)-Attr(A)$
  \item Let $W(p) = G \backslash B$
  \item Recursive call on game $G$ restricted to $A$
  \item Propagate the values on $B$.
\end{enumerate}


\paragraph{Algo 2}

Consider the Bellman operator $T$ of Puri's reduction.

\begin{enumerate}
\item Compute $v = T^n 0$.
\item Compute $A=\{ x ; v(x)=-n \}$. If $A$ is not empty, then we know that the game is won with priority $1$ from $x$. So also from $1-Attr(A)$, which we can remove from the game.
\item Compute $A=\{ x ; n-(n-1) = 1 \le v(x) \le n^2 \}$. If $A$ is not empty, then we know that the game is won with priority $2$ from $y$. So also from $0-Attr(B)$, which we can remove from the game.
\item Compute $A=\{ x ; -n^4 \le v(x) \le -n^3+(n-1)n^2=n^2 \}$....
\item Compute $A=\{ x ; n^4-(n-1)n^3 = -n^3 \le v(x) \le n^5 \}$....
\item ...
\end{enumerate}




\bibliographystyle{plain}
\bibliography{biblio.bib} 

\end{document}
