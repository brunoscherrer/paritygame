\documentclass{article}
\usepackage{fullpage}
\usepackage[utf8x]{inputenc}
\usepackage{amssymb}
\usepackage{amsthm}
\usepackage{amsmath}
\usepackage{hyperref}
\usepackage{cleveref}
\usepackage{autonum}
\usepackage{dsfont}
\usepackage{stmaryrd}

\newtheorem{theorem}{Theorem}
\newtheorem{lemma}{Lemma}
\newtheorem{remark}{Remark}
\newtheorem{corollary}{Corollary}


\def\1{{\mathds 1}}
\def\N{\mathbb N}
\def\R{\mathbb R}
\newcommand{\intset}[1]{\llbracket #1 \rrbracket}
\newcommand{\intpart}[1]{\lceil #1 \rceil}

\begin{document}

\noindent Bruno Scherrer, bruno.scherrer@inria.fr, \today.

~\\

Consider a parity game $G$ between EVEN (or Player 0) and ODD (or Player 1) on a state space $X=\{1,\dots,n\}$ with priority function $\Omega:X \to \{1,\dots,d\}$. On infinite plays, ODD wants that among the priorities that appears infinitely often, the maximal one is odd, while EVEN wants it to be even.


Consider running the $n^2$-horizon problem, and optimizing the $d^th$ dimensional vector... The largest priority $p$ that occurs at most $n$ times from some state implies that the optimal priority in $x$ is $p$. We can find the largest priority of a parity game.

\bibliographystyle{plain}
\bibliography{biblio.bib} 

\end{document}
