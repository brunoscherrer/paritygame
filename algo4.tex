\documentclass{article}
\usepackage{fullpage}
\usepackage[utf8x]{inputenc}
\usepackage{amssymb}
\usepackage{amsthm}
\usepackage{amsmath}
\usepackage{hyperref}
\usepackage{cleveref}
\usepackage{autonum}
\usepackage{dsfont}
\usepackage{stmaryrd}

\newtheorem{theorem}{Theorem}
\newtheorem{lemma}{Lemma}
\newtheorem{remark}{Remark}
\newtheorem{corollary}{Corollary}


\def\1{{\mathds 1}}
\def\N{\mathbb N}
\def\R{\mathbb R}
\newcommand{\intset}[1]{\llbracket #1 \rrbracket}
\newcommand{\intpart}[1]{\lceil #1 \rceil}

\begin{document}

\noindent Bruno Scherrer, bruno.scherrer@inria.fr, \today.

~\\

Consider a parity game $G$ between EVEN (or Player 0) and ODD (or Player 1) on a state space $X=\{1,\dots,n\}$ with priority function $\Omega:X \to \{1,\dots,d\}$. On infinite plays, ODD wants that among the priorities that appears infinitely often, the maximal one is odd, while EVEN wants it to be even.

Consider Puri's reduction to an optimal problem that makes the problem more well-defined in the sense that it implies the existence of an optimal priority: when a player loses, she loses with the smallest priority, and when she wins, it is with the higher priority possible.

Consider some parity game $G$ for which we know that the optimal parity is either $p$ or $p+1$, and the problem of identifying the regions that can be won with priority $p+1$. (If we can solve this problem in polynomial time, then any parity game can be solved iteratively in polynomial time by considering the problems where all the priorities smaller than $p$ are made equal to $p$, for successive values of $p$).

First, one can identify a region $A$ where both players want to reach and stay forever by computing the set such that $T^n 0 = 0$ in the problem where all priorities smaller than $p+1$ are made neutral.

Then, one can solve the problem on $A$ (it is a 2 priority optimal problem).
We have two regions: $B$ that is won with priority $p+1$, and $C$ that is won with priority $p$.

Reduction to a bi-partie game ? (can we stay in the original problem).


\bibliographystyle{plain}
\bibliography{biblio.bib} 

\end{document}
