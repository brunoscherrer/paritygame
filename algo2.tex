\documentclass{article}
\usepackage{fullpage}
\usepackage[utf8x]{inputenc}
\usepackage{amssymb}
\usepackage{amsthm}
\usepackage{amsmath}
\usepackage{hyperref}
\usepackage{cleveref}
\usepackage{autonum}
\usepackage{dsfont}
\usepackage{stmaryrd}

\newtheorem{theorem}{Theorem}
\newtheorem{lemma}{Lemma}
\newtheorem{remark}{Remark}
\newtheorem{corollary}{Corollary}


\def\1{{\mathds 1}}
\def\N{\mathbb N}
\def\R{\mathbb R}
\newcommand{\intset}[1]{\llbracket #1 \rrbracket}
\newcommand{\intpart}[1]{\lceil #1 \rceil}

\begin{document}

\noindent Bruno Scherrer, bruno.scherrer@inria.fr, \today.

~\\

Consider a parity game $G$ between EVEN (or Player 0) and ODD (or Player 1) on a state space $X=\{1,\dots,n\}$ with priority function $\Omega:X \to \{1,\dots,d\}$. On infinite plays, ODD wants that among the priorities that appears infinitely often, the maximal one is odd, while EVEN wants it to be even.

~ \\

Consider the lexicographic order relation on $\N^2$:
\begin{align}
  (a,b) \le (a',b') & \Leftrightarrow a \le a' \mbox{ or }(a=a' \mbox{ and }b \le b' ),
\end{align}
the corresponding min/max/+/- operators.

Consider the cost functions:
\begin{align}
  g_p = \1_{\Omega(x)=p}*(-1)^{\Omega(x)}.
\end{align}

Consider the following Bellman operators:
\begin{align}
  T_{p,\mu,\nu} (v,w) &= (v, g_p + P_{\mu,\nu} w )\\
  T_{p}(v,w) &= \max_\mu \min_\nu T_{p,\mu,\nu}(v,w)
\end{align}

Start with $v=(0,0,\dots,0)$.
\begin{itemize}
\item While there exists $x$ such that $v(x)=0$, do
  \begin{itemize}
    \item let $A=X$.
    \item Repeat:
      \begin{itemize}
      \item Consider the game reduced to $A$ and the highest priority $p$.
      \item Compute
        \begin{align}
          B = \{x ~;~ [(T_p)^n(v,0)](x)=(0,0) \}
        \end{align}
      \item If $B=\emptyset$, then set
        \begin{align}
          \forall x \in A,~ v(x)=(-1)^p
        \end{align}
        and break the repeat loop
      \item Otherwise, let $A \leftarrow B$.
      \end{itemize}
  \end{itemize}
\item Return $v$
\end{itemize}

\paragraph{Idea:} Consider priority $p$ belonging to Player $i=p \mod 2$ and a play on the sub-game $A_p$. There exists of policy for Player $i$ such that the only possibility for Player $1-i$ to avoid losing with priority $p$ on $A_p$ is to reach $A_{p-1}$ and stay in it forever. When this set is empty, it means that Player $i$ wins (with that policy) with priority $p$ on the set $A_p$, so we update the value consequently. When this set is non-empty, we know that Player $1-i$ has a policy such that Player $i$ cannot escape $A_{p-1}$, on which Player $i$ cannot win with priority $p$ (so from $A_{p-1}$, this is what Player $1-i$ wants to do).

\newpage
Decouple $v$ and $w$ (only useful on the first step) ?:

Consider the following operators:
\begin{align}
  T_{p,\mu,\nu}w & = g_p + P_{\mu,\nu} w\\
  T_{p}w & = \max_\mu \min_\nu T_{p,\mu,\nu}w\\
  U_{\mu,\nu} v & = P_{\mu,\nu}v \\
  U v & = \max_\mu \min_\nu U_{\mu,\nu}v.
\end{align}


Start with $v=(0,0,\dots,0)$.
\begin{itemize}
\item Compute $w=U^n v$.
  \begin{itemize}
  \item Let $$A=\{ x ~;~ w(x)=0 \}.$$ If $A=\emptyset$, return $w$.
    \item Repeat:
      \begin{itemize}
      \item Consider the game reduced to $A$ and the highest priority $p$ in it.
      \item Compute
        \begin{align}
          B = \{x ~;~ [(T_p)^n 0)](x)=0 \}
        \end{align}
      \item If $B=\emptyset$, then set
        \begin{align}
          \forall x \in A,~ v(x)=(-1)^p
        \end{align}
        and break the repeat loop
      \item Otherwise, let $A \leftarrow B$ and iterate.
      \end{itemize}
  \end{itemize}
\end{itemize}


Prendre la valeur $v(x)=p*(-1)^p$ ?

\bibliographystyle{plain}
\bibliography{biblio.bib} 

\end{document}
