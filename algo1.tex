\documentclass{article}
\usepackage{fullpage}
\usepackage[utf8x]{inputenc}
\usepackage{amssymb}
\usepackage{amsthm}
\usepackage{amsmath}
\usepackage{hyperref}
\usepackage{cleveref}
\usepackage{autonum}
\usepackage{dsfont}
\usepackage{stmaryrd}

\newtheorem{theorem}{Theorem}
\newtheorem{lemma}{Lemma}
\newtheorem{remark}{Remark}
\newtheorem{corollary}{Corollary}


\def\1{{\mathds 1}}
\def\G{{\cal G}}
\def\C{{\cal C}}
\def\V{{\cal V}}
\def\Ent{\mathbb N}
\def\R{\mathbb R}
\def\N{\mathfrak N}
\def\M{\mathfrak M}
\def\greed{\mathfrak G}
\newcommand{\suc}[1]{f(#1)}

\def\={\stackrel{def}{=}}
\newcommand{\intset}[1]{\llbracket #1 \rrbracket}
\newcommand{\intpart}[1]{\lceil #1 \rceil}

\begin{document}

\noindent Bruno Scherrer, bruno.scherrer@inria.fr, \today.

~\\

Consider a parity game $G$ between EVEN (or Player 0) and ODD (or Player 1) on a state space $X=\{1,\dots,n\}$ with priority function $\Omega:X \to \{1,\dots,d\}$. On infinite plays, ODD wants that among the priorities that appears infinitely often, the maximal one is odd, while EVEN wants it to be even.

~ \\

I describe below a recursive procedure to solve this problem. Then I argue that the number of recursive calls is at most $n$.

~\\

Suppose for concreteness that we have a game $G$ of which the maximal priority $p$ is even (the odd case is similar).
Consider the payoff function
\begin{align}
  \forall x,~~ g(x) = \1_{\Omega(x)=p},
\end{align}
and the Bellman operators associated to the corresponding payoff problem: for all $v \in \R^X$,
\begin{align}
  T_{\mu,\nu} v &= g + P_{\mu,\nu}v, \\
  T_{\mu} v & = \min_\nu T_{\mu,\nu} v, \\
  \tilde T_{\nu} v & = \max_\mu T_{\mu,\nu} v, \\
   T v &= \max_\mu \min_\nu T_{\mu,\nu}v = \max_\mu T_\mu v = \min_\nu \tilde T_\nu v,
\end{align}
where $P_{\mu,\nu}$ is the deterministic transition matrix when players use the (positional) strategies $\mu$ and $\nu$.

The introduction of the payoff game is motivated by the following observations:
\begin{lemma}
  In the parity game $G$,
  \begin{enumerate}
  \item The set of states from which EVEN can force ODD to visit states with priority $p$ infinitely often (and thus win) is $\{ x ~;~ [T^{n^2}0](x) \ge n \}$ ;
  \item The set of states from which ODD can force EVEN to \emph{only} visit states with priorities lower than $p$ is $\{ x ~;~ [T^k 0](x)=0 \}$ for any $k \ge n$.
  \item From any state $x$ such that $T^k 0(x)>0$, EVEN has a strategy that allows to visit at least one state with priority $p$ in the next $k$ steps (and consequently, if ODD can force EVEN to visit states with priorities $p$ only a finite number times, then the play must reach $C$ in finite time and stay in $C$ forever). 
  \end{enumerate}
\end{lemma}
\begin{proof}  
  Let us begin by the first item. If from state $x$, EVEN can force ODD to visit states with priority $p$ infinitely often, EVEN can force to visit such states at least once every $n$ steps; on a trajectory of length $n^2$, then states with priority $p$ are visited at least $n$ times. And thus on the payoff game, $[T^{n^2}0](x) \ge n$. If from state $x$, ODD can prevent EVEN from visiting states with priority $p$ infinitely often, this means that ODD has a strategy such that against any policy of EVEN, there is a path of length at most $n-1$ followed by a trajectory that only visits states with priority smaller than $p$. On the payoff game, this means that $[T^{n^2}0](x) < n$.

  Now let us consider the second item.
  First (the easier part), assume that for some state $x$, ODD has a strategy $\nu$ such that whatever EVEN does, the trajectory only visits states with priorities smaller than $p$. Then, in the $m$-horizon payoff game for any $m \ge n$, this strategy forces the trajectory to only visit states with payoff $0$, and thus:
  \begin{align}
   0 \le [T^m 0](x) \le [(\tilde T_{\nu})^m 0](x) \le 0,
  \end{align}
  i.e. $T^m 0(x)=0$.

  Now, assume that we have a state $x$ such that $T^m(x)=0$ with $m \ge n$. Let $(\nu_1,\dots,\nu_m)$ be any sequence of stratregies such that
  \begin{align}
  T^m 0 = \tilde T_{\nu_1} \dots \tilde T_{\nu_m} 0.
  \end{align}
  Against any (positional) strategy $\mu$ of EVEN, consider the trajectory $(x_0=x,x_1,\dots,x_n)$ induced if ODD plays with $\nu_1,\dots,\nu_n$. There necessarily exists $0 \le i<j\le n$ and a state $y \in X$ such that $x_i=x_j=y$. We have
  \begin{align}
    [T_{\mu,\nu_1} \dots T_{\mu,\nu_{i-1}} T_{\mu,\nu_i} \dots T_{\mu,\nu_{j-1}} T^{m-j}0](x) \le [T^m 0](x) = 0.
  \end{align}
  Since all the payoffs are non-negative, we deduce that
  \begin{align}
    [T^{m-j}0](y)=0, \\
    [T_{\mu,\nu_i} \dots T_{\mu,\nu_{j-1}} 0](y)=0, \\
    [T_{\mu,\nu_1} \dots T_{\mu,\nu_{i-1}} 0](x)=0.
  \end{align}
  Thus, if ODD plays the infinite-horizon policy $\nu_1,\dots,\nu_{i-1}(\nu_i \dots \nu_{j-1})^\infty$ against EVEN playing $\mu$, then the infinite-horizon payoff of this play is $0$. In $G$, this means that from $x$, the priorities visited are all smaller than $p$.

  The final item is obvious by considering the strategy $\mu_1,\dots,\mu_k$ such that
  \begin{align}
    T^k 0 = T_{\mu_1} \dots T_{\mu_k} 0.
  \end{align}
\end{proof}

\paragraph{Step 1.}
Compute the set
\begin{align}
  A = \{ x ~;~ [T^{n^2} 0](x) \ge n \}.
\end{align}
From any state of $A$, we know (by the Lemma) that the game is won by EVEN. If $A=X$, EVEN wins from all states, and we can terminate. Otherwise define $B=X \backslash A$ and proceed to Step 2.

\paragraph{Step 2.}
Consider the game $G_B$ restricted to $B$. For this game and its associated payoff game, compute
\begin{align}
C = \{ x ~;~ [T^n0](x) = 0 \}.
\end{align}
On $G_B$, ODD can prevent EVEN from  visiting states with priorities $p$ infinitely often. Thus, necessarily $C \neq \emptyset$. Consider $G_C$ restriced to $C$. Solve recursively $G_C$ (observe that this game only has priorities smaller than $p$).

Let $D \subset C$ be the states of $G_C$ that are won by ODD.

If $D = \emptyset$, then the game $G_B$ is also completely known by EVEN (indeed, either ODD allows to force EVEN to reach and stay forever in $C$, and loses, or the play visits states of $B \backslash C$ infinitely often, from which EVEN can always reach a state with priority $p$). In such a case, we deduce that EVEN wins in $G_B$ from all states, and we can terminate. 

If $D \neq \emptyset$, compute $E=1-Attr(D)$ in the game $G_B$. We know that in $G_B$, ODD can win from any state of $E$.

If $E=B$, we can terminate. Otherwise, if $E \neq B$, then we loop to the beginning of Step 2 with $B$ replaced by $B \backslash E$ (observe that from states of $E \neq B$, it is the choices of EVEN that prevent ODD from reaching $D$, so ODD can still prevent EVEN from visiting states with priority $p$ infinitely often). 

~\\

To conclude, let us bound the number of calls $C(n)$ for a game of size $n$. Writing
\begin{align}
  I_c = \{ 1 \le k_1, \dots, k_c \le n ~;~ \sum_{i=1}^c k_c=n \},
\end{align}
We have $C(1)=1$ and
\begin{align}
  C(n) \le \max_c \max_{k_1,\dots,k_c} \sum_c C(k_c).
\end{align}
It is easy to prove by induction on $n$ that $C(n)\le n$.




\bibliographystyle{plain}
\bibliography{biblio.bib} 

\end{document}
